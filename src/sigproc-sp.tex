\documentclass{acm_proc_article-sp}
\makeatletter
\def\@copyrightspace{\relax}
\makeatother
\usepackage{url}
\begin{document}

\title{Empirical Comparison of Supervised Learning Algorithms}
\subtitle{[CS269 Spring 2013 Project1 Report]}

\numberofauthors{1}
\author{
Chenguang Shen\\
       \affaddr{Computer Science Department}\\
       \affaddr{University of California, Los Angeles}\\
       \affaddr{Los Angeles, CA 90095}\\
       \email{cgshen@cs.ucla.edu}
}

\date{12 December 2012}

\maketitle

\section{Summary}
In this project I compare the performance of three supervised
machine learning algorithms
(random forest, SVM, and boosted tree) on three binary classification
problems. The data sets are retrieved from the public UCI machine
learning repository \cite{uci}. The experimental setting generally
follows the work by Caruana et al. \cite{Caruana:2006},
and result is consistent with their observation with minor differences.

In this report I will introduce the detailed setting of all my
experiments, and provide comparison with the result in \cite{Caruana:2006}.
For the difference in results, I will try to explain the
possible reasons.

\section{Methodology}
\subsection{Data Sets}
I have chosen 3 data sets from the UCI data set \cite{uci}, including
ADULT, COV\_TYPE and LETTER. The basic information of these 3 data sets
are shown in Table 1.

The original ADULT data set on the UCI repository has 14 features,
and is a binary classification problem. The features describes basic
information of a person, and the label shows whether his/her annual
earning is higher than 50K. However, 8 of the 14
features are multi-class categorical strings, therefore it is not 
optimized for some of the classifiers. In order to get better
performance, I used an optimized version of the ADULT data set
by Platt et al. \cite{Platt:1999}, where
continuous features are discretized into quantiles, and 
each quantile is represented by a binary feature. Also, a categorical 
feature with m categories is converted to m binary features.
There revised data set has 123 binary features with value \{0, 1\}.

The COV\_TYPE data set is a multi-class classification problem.
It is used to predict forest cover type from its properites.
It has been converted to a binary problem by treating the 
largest class as the positive and the rest as negative. I addition,
all 14 features are scaled to [0..1] using the LIBSVM \cite{CC01a}
scaling tool.

The LETTER data set is also multi-class classification problem.
The objective is to identify each of a large number of 
black-and-white rectangular pixel displays as one of 
the 26 capital letters in the English alphabet. In order to convert
it to a binary classification problem, LETTER.p1 treats �O� 
as positive and the remaining 25 letters as negative, 
yielding a very unbalanced problem. LETTER.p2 uses letters 
A-M as positives and the rest as negatives, 
yielding a well balanced problem. The data in both case are not scaled.

Since the LETTER data set is evolved into LETTER.p1 and
LETTER.p2, there are actually four problems in total. For each
data set, I choose the training and testing data as a whole 
form the data set file, and randomly select 5000 samples as
training set. The random selection process is done by
Python's {\em random.shuffle()} method.

\begin{table}
    \center
    \begin{tabular}{|cccc|}
    \hline
    PROBLEM   & \# ATTR & TRAIN SIZE & TEST SIZE \\ \hline
    ADULT     & 123     & 5000       & 25000     \\
    COV\_TYPE  & 54      & 5000       & 25000     \\
    LETTER.P1 & 16      & 5000       & 14000     \\
    LETTER.P2 & 16      & 5000       & 14000     \\ \hline
    \end{tabular}
    \caption {Summary of Data Sets}
\end{table}



\subsection{Learning Algorithms}
I implement all three algorithms in Python, with the help of 
scikit-learn package \cite{scikit-learn}. The parameter setting
of each of the classifiers follows \cite{Caruana:2006},
and is discussed below:

{\bf Random Forest}\\
The forests used in this project have 1024 trees. 
The size of the feature set considered at each split 
is 1, 2, 4, 6, 8, 12, 16 or 20. When this size exceeds the total
number of features in a specific data set, I just use
the maximum number of features

{\bf SVM}\\
I use the following kernels: linear, polynomial degree 2 \& 3, 
radial with width \{0.001, 0.005, 0.01, 0.05, 0.1, 0.5, 1, 2\}. 

{\bf Boosted Tree}\\
I used scikit-learn's {\em GradientBoostingClassifier} implementation
for boosted trees. In each stage a binary regression trees 
are fit on the negative gradient of the binomial 
deviance loss function.
Boosting can overfit, so I consider boosted trees
after 2, 4, 8, 16, 32, 64, 128, 256, 512, 1024 
and 2048 steps of boosting.

Note that I am not using any calibration method for these
classifiers.

\subsection{Performance Metrics}
I use ACC, FSC, ROC, and APR as performance metrics for the three
algorithms:

{\bf ACC} \\
Accuracy of the classification.

{\bf FSC} \\
F-score of the classification (treating the binary classification
problem as a true-false problem). I use F1 score in this project.

{\bf ROC}\\
Area under the ROC (Receiver Operating Characteristic) curve, 
treating the binary classification problem as a true-false problem.

{\bf APR}\\
Average precision of the classification, 
treating the binary classification problem as a true-false problem.

\section{Performance by Metric}
To test the performance of each algorithm, I used the same trial 
method as in \cite{Caruana:2006}. For each algorithm, I conduct
a 5-fold cross validation and obtain 5 trials. That is, for the 4000
training samples and 1000 testing samples in each cross validation,
I will try to train the classifier with all possible parameters,
and use the 1000 testing samples to obtain the best classifier. 
The metric for selecting best classifier is just aver score
over the 4 metrics, and no calibration method is used.
Then the best trained classifier is used to test the large testing set.

In this process, I get the mean metric score for each classifier 
over all 3 problems, both for the cross-validation, and for the larger
testing set. These two kinds of score are listed separately.

\subsection{Cross-validation scores}
Table 2 shows the 5-fold CV scores for each learning 
algorithm by metric, which is the average over four
problems. This is the score of applying one classifier
with its best parameter setting on the small test set
(1000 samples) in each 5-fold cross-validation.
Random forest has highest scores on every
metric, and is ranked as the best algorithm among these three.
Boosted tree scores better than SVM on accuracy and precision,
but is worse on area under ROC curve and f-score. On average
boosted tree scores a little better than SVM. 

\subsection{Test scores}
Table 3 shows the test scores for each learning 
algorithm by metric, which is the average over four
problems. This is the average score of using the best classifier
trained from 5-fold cross-validation to test on the large
test data set. In Table 3, random forest has the best accuracy
and precision, while SVM has the best are under ROC curve, and f-score.
On average random forest is the best among these three, while
SVM and boosted tree are ranked in 2nd and 3rd place, respectively.
Note that the score difference here between these three algorithms
are pretty small.

\subsection{Analysis}
In paper \cite{Caruana:2006}, similar result on testing set
is presented, which is summarized in Table 4. Note that 
because I am not using any calibration method here, I have chosen
the corresponding result from \cite{Caruana:2006} where no calibration
is used. The result in Table 4 should be compared with the Table 3 here.

In both Table 3 and Table 4, random forest is ranked as the
best algorithm. However, the rank of SVM and boosted tree
is reversed in my result (Table 3). This is probably because
I am using scikit-learn's {\em GradientBoostingClassifier} 
implementation, where the type of tree is not clear. However, 
in \cite{Caruana:2006} the author consider more difference kinds
of trees, such as BAYES, ID3, CART, CART0, C4, MML, and SMML.
In each 5-fold cross-validation stage the best tree will be
selected out of these types. Therefore it is reasonable to 
assume some other types of decision tree will have better
performance than the tree used by {\em GradientBoostingClassifier}.
In addition, in Table 4 boosted tree has best scores in APR
and FSC, but not in Table 3. This is probably because of the same 
reason stated above.

In general, the MEAN numbers in Table 3 are smaller than 
those in Table 4. This is correct because I am not 
normalizing the scores in my result. However, there is some 
exceptions, such as the accuracy for all three classifiers.
One reason is that I am not using exactly the same data set
as in \cite{Caruana:2006}. For ADULT, I am using the optimized
data set instead of the original one. For COV\_TYPE, I am
using the scaled version of the original dataset. These two
optimized version of data sets have resulted in the improved
performance in accuracy of my classifier, especially for SVM,
which is known to perform poorly with non-scaled data set.

\begin{table*}
	\centering
    \begin{tabular}{|c|cccc|c|}
    \hline
    MODEL & ACC   & ROC   & APR   & FSC   & MEAN  \\ \hline
    RF     & 0.908 & 0.871 & 0.875 & 0.839 & 0.873 \\
    BST-DT & 0.901 & 0.862 & 0.866 & 0.830 & 0.865 \\
    SVM    & 0.896 & 0.868 & 0.857 & 0.834 & 0.864 \\ \hline
    \end{tabular}
    \caption {5-fold CV scores for each learning algorithm by metric (average over four problems)}
\end{table*}

\begin{table*}
\centering
    \begin{tabular}{|c|cccc|c|}
    \hline
    MODEL & ACC   & ROC   & APR   & FSC   & MEAN  \\ \hline
    RF     & 0.890 & 0.836 & 0.844 & 0.794 & 0.841 \\
    SVM    & 0.884 & 0.843 & 0.833 & 0.801 & 0.840 \\
    BST-DT & 0.885 & 0.835 & 0.839 & 0.794 & 0.838 \\ \hline
    \end{tabular}
    \caption {Testing scores for each learning algorithm by metric (average over four problems)}
\end{table*}

\begin{table*}
	\centering
    \begin{tabular}{|c|cccc|c|}
    \hline
    MODEL & ACC   & ROC   & APR   & FSC   & MEAN  \\ \hline
    RF     & 0.872 & 0.951 & 0.931 & 0.934 & 0.922 \\
    BST-DT & 0.834 & 0.963 & 0.938 & 0.816 & 0.888 \\
    SVM    & 0.817 & 0.938 & 0.899 & 0.804 & 0.864 \\ \hline
    \end{tabular}
    \caption{Normalized performance by metric result in \cite{Caruana:2006}}
\end{table*}


\section{Performance by Problem}
Table 5 and Table 6 show the score for each algorithm
on each of the 3 test problems in 5-fold validation and test,
respectively.

\subsection{Cross-validation scores}
Table 5 shows the 5-fold cross-validation scores for each learning 
algorithm by problem, which is the average over four
metrics. For COV\_TYPE and LETTER.p1, random forest
has the best average performance. For ADULT and LETTER.p2, SVM
has the best average performance.

\subsection{Test scores}
Table 6 shows the test scores for each learning 
algorithm by metric, which is the average over four
problems. For COV\_TYPE, random forest
has the best average performance. For ADULT, LETTER.p1, 
and LETTER.p2, SVM has the best average performance.

\subsection{Analysis}
The corresponding result on performance-by-problem for testing set
in \cite{Caruana:2006} is shown in Table 7. Note that because
they are using more metrics (in total 8 metrics) than I do, it
is not particularly meaningful to compare the result by each
problem. However, we can still find some clue about the difference.

The most important observation is that SVM is performing better
than boosted tree in my experiment using the big testing set in Table 6,
while in Table 7 SVM performs much worse than boosted tree. Once again
I concludes the reason for this is because I am using some optimized
data set, which could improve the performance of SVM significantly.
SVM generally does not perform well when data is not scaled. By using 
the optimized ADULT data set and the scaled COV\_TYPE data set,
the performance of SVM is significantly improved.

\begin{table*}
	\centering
    \begin{tabular}{|c|cccc|c|}
    \hline
    MODEL & COV\_TYPE & ADULT & LETTER.p1 & LETTER.p2 & MEAN  \\ \hline
    RF     & 0.828    & 0.736 & 0.972     & 0.958     & 0.874 \\
    BST-DT & 0.800    & 0.740 & 0.961     & 0.957     & 0.864 \\
    SVM    & 0.774    & 0.746 & 0.972     & 0.963     & 0.863 \\ \hline
    \end{tabular}
    \caption {5-fold CV scores of each learning algorithm by problem (averaged over for metrics)}
\end{table*}

\begin{table*}
\centering
    \begin{tabular}{|c|cccc|c|}
    \hline
    MODEL & COV\_TYPE & ADULT & LETTER.p1 & LETTER.p2 & MEAN  \\ \hline
    RF     & 0.818    & 0.713 & 0.931     & 0.902     & 0.841 \\
    SVM    & 0.772    & 0.742 & 0.933     & 0.912     & 0.839 \\
    BST-DT & 0.794    & 0.733 & 0.925     & 0.901     & 0.838 \\ \hline
    \end{tabular}
    \caption {Testing scores of each learning algorithm by problem (averaged over for metrics)}
\end{table*}

\begin{table*}
\centering
    \begin{tabular}{|c|cccc|c|}
    \hline
    MODEL & COV\_TYPE & ADULT & LETTER.p1 & LETTER.p2 & MEAN  \\ \hline
    RF     & 0.876    & 0.946 & 0.883     & 0.922     & 0.907 \\
    BST-DT & 0.874    & 0.842 & 0.875     & 0.913     & 0.876 \\ 
    SVM    & 0.696    & 0.819 & 0.731     & 0.860     & 0.776 \\\hline
    \end{tabular}
    \caption {Normalized performance by problem result in \cite{Caruana:2006}}
\end{table*}

\section{Bootstrap Analysis}
Similar to the \cite{Caruana:2006} paper, I also finish the bootstrap
analysis. Since I only have 4 problems, I did not do sampling with
replacement on the problem set. Instead, I run each classification
algorithm 30 times with each of the 4 problems, using sampling with
replacement to get the input data set. For each of the run, I also use
5-fold cross-validation to obtain 5 trials, and use the best trained
classifier to test on the large test set. I only run each algorithm 
30 times on each problem because it is pretty expensive to increase
the running times.

The result of the bootstrap analysis is shown in Table 8. The rank
of the 3 algorithms (RF, SVM, BST-DT) is consistent with the result
in Table 3 and Table 6, showing that the results in the previous 
performance-by-metric and performance-by-problem experiments are
stable. Since the paper \cite{Caruana:2006} is using the best 
trained classifier for each algorithm, it is not appropriate to compare
it with the result here, since I am not using all parameters and there
is not calibration in my algorithm.

\begin{table}
	\centering
    \begin{tabular}{|c|ccc|}
    \hline
    MODEL & 1ST   & 2ND   & 3RD   \\ \hline
    RF     & 0.392 & 0.200 & 0.408 \\
    SVM    & 0.383 & 0.283 & 0.333 \\
    BST-DT & 0.225 & 0.517 & 0.259 \\ \hline
    \end{tabular}
    \caption {Bootstrap analysis of overall rank by mean performance across problems and metrics}\end{table}

\section{Conclusion}
Generally, the results in this project are consistent with
the result shown by Caruana et al. \cite{Caruana:2006}. Without
any calibration, random forest always has the best overall
performance. In addition, random forest has the simplest 
algorithm of all these three ones. However, it also takes the
most time to train. 

The inconsistency of my result compared to the Caruana paper
comes from three aspects:
\begin{enumerate}
\item Due to the time limit, I am not using all parameters
described in the Caruana paper, and resulting bad performance
of boosted tree.
\item I am not doing calibration for all algorithms.
\item I am using optimized and scaled data sets for ADULT
and COV\_TYPE, resulting improved performance of SVM.
\end{enumerate}

\bibliographystyle{abbrv}
\bibliography{sigproc}  
\balancecolumns

\end{document}
